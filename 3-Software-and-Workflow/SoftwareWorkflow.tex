% !TEX TS-program = pdflatex
% !TEX encoding = UTF-8 Unicode

% This file is a template using the "beamer" package to create slides for a talk or presentation
% - Giving a talk on some subject.
% - The talk is between 15min and 45min long.
% - Style is ornate.

% MODIFIED by Jonathan Kew, 2008-07-06
% The header comments and encoding in this file were modified for inclusion with TeXworks.
% The content is otherwise unchanged from the original distributed with the beamer package.

\documentclass{beamer}


% Copyright 2004 by Till Tantau <tantau@users.sourceforge.net>.
%
% In principle, this file can be redistributed and/or modified under
% the terms of the GNU Public License, version 2.
%
% However, this file is supposed to be a template to be modified
% for your own needs. For this reason, if you use this file as a
% template and not specifically distribute it as part of a another
% package/program, I grant the extra permission to freely copy and
% modify this file as you see fit and even to delete this copyright
% notice. 


\mode<presentation>
{
  \usetheme{Berkeley}
  % or ...

  \setbeamercovered{transparent}
  % or whatever (possibly just delete it)
}

\usepackage{tikz}
\usepackage{graphicx}
\graphicspath{{C:/Users/garret/GoogleDrive/CEGA/CEGA-Programs/BITSS/Supporting-Material-Slides/APHRC/Images/}}
\usepackage[english]{babel}
% or whatever

\usepackage[utf8]{inputenc}
% or whatever

\usepackage{times}
\usepackage[T1]{fontenc}
% Or whatever. Note that the encoding and the font should match. If T1
% does not look nice, try deleting the line with the fontenc.


\title[Short Paper Title] % (optional, use only with long paper titles)
{Software and Workflow for Reproducible Research}

\subtitle
{Discussion} % (optional)

\author[Christensen] % (optional, use only with lots of authors)
{Garret Christensen\inst{1}}
% - Use the \inst{?} command only if the authors have different
%   affiliation.

\institute[UC Berkeley, Berkely Initiatiative for Transparency in the Social Sciences] % (optional, but mostly needed)
{
  \inst{1}%
  UC Berkeley: Berkely Initiatiative for Transparency in the Social Sciences\\
  Berkeley Institute for Data Science
}
% - Use the \inst command only if there are several affiliations.
% - Keep it simple, no one is interested in your street address.

\date[Short Occasion] % (optional)
{APHRC, Summer 2015}

\subject{Talks}
% This is only inserted into the PDF information catalog. Can be left
% out. 

\pgfdeclareimage[height=2cm]{university-logo}{C:/Users/garret/GoogleDrive/CEGA/CEGA-Programs/BITSS/Supporting-Material-Slides/APHRC/Images/BITSSlogo.png}
 \logo{\pgfuseimage{university-logo}}

% Delete this, if you do not want the table of contents to pop up at
% the beginning of each subsection:
%\AtBeginSubsection[]
%{
%  \begin{frame}<beamer>{Outline}
%   \tableofcontents[currentsection,currentsubsection]
%  \end{frame}
%}


% If you wish to uncover everything in a step-wise fashion, uncomment
% the following command: 

%\beamerdefaultoverlayspecification{<+->}

\begin{document}

% Since this a solution template for a generic talk, very little can
% be said about how it should be structured. However, the talk length
% of between 15min and 45min and the theme suggest that you stick to
% the following rules:  

% - Exactly two or three sections (other than the summary).
% - At *most* three subsections per section.
% - Talk about 30s to 2min per frame. So there should be between about
%   15 and 30 frames, all told.

\begin{frame}
  \titlepage
\end{frame}

\begin{frame}{Outline}
  \tableofcontents
  % You might wish to add the option [pausesections]
\end{frame}


\section{Introduction}
\begin{frame}{Reproducibility \& Transparency}
\begin{itemize}
\item What are problems associated with reproducibility?
\item What are solutions to these problems?
\item What are practical tools to implement these solutions?
\end{itemize}
\end{frame}
%%%%%%%%%%%%%%%%%%%%%%%%%%%%%%%%%%%%%%%%%%%%%%%%%%%%%%%%%%%%%%%%%%
\section{Problems}
\begin{frame}{Problems}
 \begin{itemize}
 \item Publication bias (see previous talk)
 \item Specification Searching (see previous talk)
 \item Data not available
 \item Code not available/unintelligible
 \item Code and data cannot reproduce original results
 \end{itemize}
\end{frame}

 
\subsection{Irreproducible Workflow}
 \begin{frame}{Irreproducible Workflow}
 \begin{itemize}
 \item
  Even with the original authors' help, you can't get the data to reproduce the published results. Or you just can't find the data to begin with. 
  \item \textit{Journal of Money, Credit, and Banking} Project. \href{http://www.jstor.org/stable/1806061}{(Dewald et al., AER 1986)}
   \item Martin Feldstein on Social Security and private savings, Reinhart and Rogoff on debt and GDP growth.
 \end{itemize} 
 \end{frame}
%%%%%%%%%%%%%%%%%%%%%%%%%%%%%%%%%%%%%%%%%%%%%%%%%%%%%%%%%%%%%%%%%%%%

\section{Solutions}
\begin{frame}{Solutions}
\begin{itemize}[<+->]
\item Study Registry (see previous talk)
\item Pre-Analysis Plan (see previous talk)
\item Reproducible Workflow
\end{itemize}
\end{frame}

%%%%%%%%%%%%%%%%%%%%%%%%%%%%%%%%%%%%%%%%%%%%%%%%%%%%%%%%%%%%%%%%%%%%%

\subsection{Workflow}
\begin{frame}{Reproducible Workflow}
 \begin{itemize}
 \item Literate Programing 
 \item R Markdown and \href{http://rstudio.com}{R Studio} to write dynamic documents.
 \item Version control with \href{http://www.github.com}{Github} or \href{http://osf.io}{OSF}. 
 \item Data Sharing
 \begin{itemize}
 \item \href{http://www.thedata.org}{Harvard's Dataverse}
 \end{itemize}
\end{itemize}
\end{frame}
%%%%%%%%%%%%%%%%%%%%%%%%%%%%%%%%%%%%%%%%%%%%%%%%%%%%%%%%%%%%%%%%%%%%%
\subsection{Literate Programming}
\begin{frame}{Literate Programming}
First, you should be \textit{programming}. Working in Excel is not reproducible.
Reinhart and Rogoff. If you are using SPSS, there is a `syntax' command to record all the commands you run. Similarly in Stata, commandlog.

Better is to write scripts. R, Stata, SAS, Python, whatever. Open source has some advantages (being free, for one) but you're going to use what everyone in your field uses.

Second, literate programming. Write code to be read by a human being, with the code for the computer secondary.
\end{frame}
%%%%%%%%%%%%%%%%%%%%%%%%%%%%%%%%%%%%%%%%%%%%%%%%%%%%%%%%%%%%%%%%%%%%%
\subsection{Version Control}
\begin{frame}{Version Control}
Use version control.

DCVS distributed version control system

gentzkow and shapiro's RA guide

\end{frame}

{ % all template changes are local to this group.
    \setbeamertemplate{navigation symbols}{}
    \begin{frame}[plain]
        \begin{tikzpicture}[remember picture,overlay]
            \node[at=(current page.center)] {
                \includegraphics[height=\paperheight]{github-logo-transparent.jpg}
            };
        \end{tikzpicture}
     \end{frame}

    \begin{frame}[plain]
        \begin{tikzpicture}[remember picture,overlay]
            \node[at=(current page.center)] {
                \includegraphics[height=\paperheight]{OSFnow.PNG}
            };
        \end{tikzpicture}
     \end{frame}
 
    \begin{frame}[plain]
        \begin{tikzpicture}[remember picture,overlay]
            \node[at=(current page.center)] {
                \includegraphics[height=\paperheight]{OSFsoon.PNG}
            };
        \end{tikzpicture}
     \end{frame}
}

\begin{frame}
GitHub and OSF Examples
\end{frame}
%%%%%%%%%%%%%%%%%%%%%%%%%%%%%%%%%%%%%%%%%%%%%%%%%%%%%%%%%%%%%%%%%%%%%
\subsection{Dynamic Documents}
\begin{frame}{Dynamic Documents}
A dynamic document includes your data, code, analysis, and output all in one place. Fully automated, you can guarantee no mistakes from copying and pasting.

Do this with R Markdown and R Studio or Ketchup in Stata.
\end{frame}

\begin{frame}
R Studio Example

Stata Example
\end{frame}
%%%%%%%%%%%%%%%%%%%%%%%%%%%%%%%%%%%%%%%%%%%%%%%%%%%%%%%%%%%%%%%%%%%%%
\subsection{Data Sharing}
\begin{frame}{Data Sharing}
Make all your data and code publicly available.

Put it in a place where people can find it. 

 For APHRC that might be the APHRC repository.
\end{frame} 
%%%%%%%%%%%%%%%%%%%%%%%%%%%%%%%%%%%%%%%%%%%%%%%%%%%%%%%%%%%%%


\section{Conclusion}
\begin{frame}{Conclusion}
Simple tools exist to help you transparently and reproducibly take your research from beginning to end. 
\begin {itemize}
\item Open Science Framework
\item Trial Registries
\item Version Control
\item Dynamic Documents
\item Trusted Public Data Archive
\end{itemize} 
\vspace{0.25in}
Read more in my \href{http://github.com/garretchristensen/manual}{\textit{Manual of Best Practices in Transparent Social Science Research}} on GitHub.
\end{frame}


\end{document}


