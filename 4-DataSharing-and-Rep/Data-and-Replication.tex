% !TEX TS-program = pdflatex
% !TEX encoding = UTF-8 Unicode

% This file is a template using the "beamer" package to create slides for a talk or presentation
% - Giving a talk on some subject.
% - The talk is between 15min and 45min long.
% - Style is ornate.

% MODIFIED by Jonathan Kew, 2008-07-06
% The header comments and encoding in this file were modified for inclusion with TeXworks.
% The content is otherwise unchanged from the original distributed with the beamer package.

\documentclass{beamer}


% Copyright 2004 by Till Tantau <tantau@users.sourceforge.net>.
%
% In principle, this file can be redistributed and/or modified under
% the terms of the GNU Public License, version 2.
%
% However, this file is supposed to be a template to be modified
% for your own needs. For this reason, if you use this file as a
% template and not specifically distribute it as part of a another
% package/program, I grant the extra permission to freely copy and
% modify this file as you see fit and even to delete this copyright
% notice. 


\mode<presentation>
{
  \usetheme{Berkeley}
  % or ...

  \setbeamercovered{transparent}
  % or whatever (possibly just delete it)
}

\usepackage{tikz}
\usepackage{graphicx}
\graphicspath{{C:/Users/garret/GoogleDrive/CEGA/CEGA-Programs/BITSS/Supporting-Material-Slides/APHRC/Images/}}
\usepackage[english]{babel}
% or whatever

\usepackage[utf8]{inputenc}
% or whatever

\usepackage{times}
\usepackage[T1]{fontenc}
% Or whatever. Note that the encoding and the font should match. If T1
% does not look nice, try deleting the line with the fontenc.


\title[Short Paper Title] % (optional, use only with long paper titles)
{Data Sharing and Replication}

\subtitle
{Enabling Reproducible Research} % (optional)

\author[Christensen] % (optional, use only with lots of authors)
{Garret Christensen\inst{1}}
% - Use the \inst{?} command only if the authors have different
%   affiliation.

\institute[UC Berkeley, Berkely Initiatiative for Transparency in the Social Sciences] % (optional, but mostly needed)
{
  \inst{1}%
  UC Berkeley: Berkely Initiatiative for Transparency in the Social Sciences\\
  Berkeley Institute for Data Science
}
% - Use the \inst command only if there are several affiliations.
% - Keep it simple, no one is interested in your street address.

\date[Short Occasion] % (optional)
{APHRC, Summer 2015}

\subject{Talks}
% This is only inserted into the PDF information catalog. Can be left
% out. 

\pgfdeclareimage[height=2cm]{university-logo}{C:/Users/garret/GoogleDrive/CEGA/CEGA-Programs/BITSS/Supporting-Material-Slides/APHRC/Images/BITSSlogo.png}
 \logo{\pgfuseimage{university-logo}}

% Delete this, if you do not want the table of contents to pop up at
% the beginning of each subsection:
%\AtBeginSubsection[]
%{
%  \begin{frame}<beamer>{Outline}
%   \tableofcontents[currentsection,currentsubsection]
%  \end{frame}
%}


% If you wish to uncover everything in a step-wise fashion, uncomment
% the following command: 

%\beamerdefaultoverlayspecification{<+->}

\begin{document}

% Since this a solution template for a generic talk, very little can
% be said about how it should be structured. However, the talk length
% of between 15min and 45min and the theme suggest that you stick to
% the following rules:  

% - Exactly two or three sections (other than the summary).
% - At *most* three subsections per section.
% - Talk about 30s to 2min per frame. So there should be between about
%   15 and 30 frames, all told.

\begin{frame}
  \titlepage
\end{frame}

\begin{frame}{Outline}
  \tableofcontents
  % You might wish to add the option [pausesections]
\end{frame}


\section{Introduction}
\begin{frame}{Reproducibility \& Transparency}
\begin{itemize}
\item What are problems associated with reproducibility?
\item What are solutions to these problems?
\item What are practical tools to implement these solutions?
\end{itemize}
\end{frame}
%%%%%%%%%%%%%%%%%%%%%%%%%%%%%%%%%%%%%%%%%%%%%%%%%%%%%%%%%%%%%%%%%%
\begin{frame}{Introduction}
Science advances by building on the work of others. 
\vspace{0.5in}


\begin{quote}
If I have seen further, it is by standing on the shoulders of giants
\end{quote}
\begin{flushright}
--Sir Isaac Newton, 1676
\end{flushright}
\end{frame}
%%%%%%%%%%%%%%%%%%%%%%%%%%%%%%%%%%%%

\begin{frame}{Problems}
What prevents us from building on others' work?
\begin{itemize}[<.->]
 \item Data not shared
 \item Analysis not shared
 \item Methods/protocol not shared 
\end{itemize}
\end{frame}

%%%%%%%%%%%%%%%%%%%%%%%%%%%%%%%%%%%%%%%
\begin{frame}{Solutions}
What enables us to build on others' work?
\begin{itemize}[<.->]
 \item Data shared in trusted public repository
 \item Code/Analysis shared in trusted public repository
 \item Methods/protocol follow appropriate reporting standard
 \item Also: findings/scholarly publications available (open access)
\end{itemize}
\end{frame}



%%%%%%%%%%%%%%%%%%%%%%%%%%%%%%%%%%%%%%%%%%%%%%%%%%%
\section{Project Protocol, Reporting Standards}
\begin{frame}[<.->]{Project Protocol, Reporting Standards}
 Make sure you report everything another researcher would need to replicate your research, including the exact methods.
 \pause
 
 What to report (following medicine):
\begin{itemize}
\item Find the appropriate reporting standard for your field and follow it.
\item Enhancing the QUAlity and Transparency Of health Research (\href{http://www.equator-network.org}{EQUATOR Network})
\item The most widely-adopted standard: \href{http://www.consort-statement.org}{Consolidated Standards of Reporting Trials (CONSORT)}. 
\item Standard Protocol Items: Recommendations for Interventional Trials (\href{http://www.spirit-statement.org}{SPIRIT Statement}).
\end{itemize}
\end{frame}

\begin{frame}{Project Protocol, Reporting Standards}
Where to report:

If not in the methods section of the article (of limited length), supplementary online appendix linked with article or in trusted digital repository.
\end{frame}
%%%%%%%%%%%%%%%%%%%%%%%%%%%%%%%%%%%%%%%%%%%%%%%%%%%%%%%%%%%%%%%%%%%%%%%
\section{Data Sharing}
\begin{frame}{Data Sharing}
To build on the work of others, data must be shared.

Data sharing is associated with more citations (causality unclear). \href{http://journals.plos.org/plosone/article?id=10.1371/journal.pone.0000308}{Piwowar et al. 2007} 
\end{frame}
%%%%%%%%%%%%%%%%%%%%%%%%%%%%%%%%%%%%%%%%%%%%%%%%%%%%%

\begin{frame}{Data Sharing}
History in Economics:
\begin{itemize}
\item  \href{http://www.jstor.org/stable/1806061}{Journal of Money Credit and Banking Project}: Dewald, Thursby, Anderson \textit{AER} 1986. 
\begin{itemize}
\item Low response rate to requests to share data.
\item Attempted to reproduce 9 papers, problems with all (some minor) even with help of original authors.
\end{itemize}
\end{itemize}
\end{frame}


{ % all template changes are local to this group.
    \setbeamertemplate{navigation symbols}{}
    \begin{frame}[plain]
        \begin{tikzpicture}[remember picture,overlay]
            \node[at=(current page.center)] {
                \includegraphics[width=\paperwidth]{JMCBTable1.PNG}
            };
        \end{tikzpicture}
     \end{frame}
}

%%%%%%%%%%%%%%
\begin{frame}{Data Sharing}
History in Economics:
\begin{itemize}
\item  \href{http://www.jstor.org/stable/1806061}{Journal of Money Credit and Banking Project}: Dewald, Thursby, Anderson \textit{AER} 1986. 
 \begin{itemize}
 \item Low response rate to requests to share data.
 \item Attempted to reproduce 9 papers, problems with all (some minor) even with help of original authors.
 \end{itemize}
\item \href{https://research.stlouisfed.org/publications/review/94/11/Replication_Nov_Dec1994.pdf}{A Decade After JMCB}: Anderson and Dewald, St Louis Fed 1994.
 \begin{itemize}
 \item Repeated similar experiment
 \item Similar bleak results
 \end{itemize}
 \pause
\item Verifying the Solution from a Nonlinear Solver, McCullough and Vinod, \textit{AER} 2003.
 \begin{itemize}
 \item Different software programs get you different answers.
 \item But finally change---\textit{AER} institutes data sharing requirement. \href{https://www.aeaweb.org/aer/data.php}{\beamerbutton{Policy}}
 \end{itemize} 
\end{itemize}
\end{frame}
%%%%%%%%%%%%%%


\begin{frame}{Data Sharing}
How are we doing as a discipline?
\begin{itemize}
\item \textit{AER} internal review generally positive (\href{https://www.aeaweb.org/aer/2011_Data_Compliance_Report.pdf}{Glandon 2010})

\item Many, including McCullough, still skeptical of the ability to reproduce (\href{http://econjwatch.org/articles/got-replicability-the-journal-of-money-credit-and-banking-archive}{Econ Journal Watch, 2007})

\item Though \textit{AER}, all AEA, and other top journals have a good policy, enforcement is limited, and shared data is often only the ``analysis'' data instead of raw data, and \textit{QJE} has no policy whatsoever.

\item A study by the Replication Network shows that fewer than 27 journals regularly publish data, only 10 explicitly state they publish replications. (\href{http://econjwatch.org/articles/replications-in-economics-a-progress-report}{Duvendack et al 2015})

\end{itemize}
\end{frame}

%%%%%%%%%%%%%%%%%%%%%

\begin{frame}{Data Sharing}
Why share your data in a trusted public repository?
\begin{itemize}[<.->]
\item
Find the appropriate repository: \url{http://www.re3data.org/}
\item
Repositories will last longer than your own website.
\item
Repositories are more easily searchable by other researchers.
\item
Repositories will store your data in a non-proprietary format that won't become obsolete.
\end{itemize}
\end{frame}

\begin{frame}{Data Sharing}
Data repositories

Harvard's Dataverse

Check the journal
\end{frame}

\begin{frame}{APHRC Repository}
APHRC has its own repository. You should use it.
\end{frame}


%%%%%%%%%%%%%%%%%%%%%%%%%%%%%%%%%%%%%%
\section{Replication}
\begin{frame}{Replication}
Define Replication

Michael Clemens
\end{frame}


\begin{frame}{Replication}
How to do a replication
\end{frame}


\begin{frame}{Replication}
Encourage it, make it publishable
\end{frame}

%%%%%%%%%%%%%%%%%%%%%%%%%%%%%%%%%%%%%%
\section{Conclusion}
\begin{frame}{Conclusion}
\begin{itemize}
\item Share your data
\item Replicate
\end{itemize}
\end{frame}

\end{document}


