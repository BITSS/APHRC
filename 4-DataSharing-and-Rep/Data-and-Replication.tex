% !TEX TS-program = pdflatex
% !TEX encoding = UTF-8 Unicode

% This file is a template using the "beamer" package to create slides for a talk or presentation
% - Giving a talk on some subject.
% - The talk is between 15min and 45min long.
% - Style is ornate.

% MODIFIED by Jonathan Kew, 2008-07-06
% The header comments and encoding in this file were modified for inclusion with TeXworks.
% The content is otherwise unchanged from the original distributed with the beamer package.

\documentclass{beamer}


% Copyright 2004 by Till Tantau <tantau@users.sourceforge.net>.
%
% In principle, this file can be redistributed and/or modified under
% the terms of the GNU Public License, version 2.
%
% However, this file is supposed to be a template to be modified
% for your own needs. For this reason, if you use this file as a
% template and not specifically distribute it as part of a another
% package/program, I grant the extra permission to freely copy and
% modify this file as you see fit and even to delete this copyright
% notice. 


\mode<presentation>
{
  \usetheme{Berkeley}
  % or ...

  \setbeamercovered{transparent}
  % or whatever (possibly just delete it)
}

\usepackage{tikz}
\usepackage{graphicx}
\graphicspath{{C:/Users/garret/GoogleDrive/CEGA/CEGA-Programs/BITSS/Supporting-Material-Slides/APHRC/Images/}}
\usepackage[english]{babel}
% or whatever

\usepackage[utf8]{inputenc}
% or whatever

\usepackage{times}
\usepackage[T1]{fontenc}
% Or whatever. Note that the encoding and the font should match. If T1
% does not look nice, try deleting the line with the fontenc.


\title[Short Paper Title] % (optional, use only with long paper titles)
{Data Sharing and Replication}

\subtitle
{Transparency} % (optional)

\author[Christensen] % (optional, use only with lots of authors)
{Garret Christensen\inst{1}}
% - Use the \inst{?} command only if the authors have different
%   affiliation.

\institute[UC Berkeley, Berkely Initiatiative for Transparency in the Social Sciences] % (optional, but mostly needed)
{
  \inst{1}%
  UC Berkeley: Berkely Initiatiative for Transparency in the Social Sciences\\
  Berkeley Institute for Data Science
}
% - Use the \inst command only if there are several affiliations.
% - Keep it simple, no one is interested in your street address.

\date[Short Occasion] % (optional)
{APHRC, Summer 2015}

\subject{Talks}
% This is only inserted into the PDF information catalog. Can be left
% out. 

\pgfdeclareimage[height=2cm]{university-logo}{C:/Users/garret/GoogleDrive/CEGA/CEGA-Programs/BITSS/Supporting-Material-Slides/APHRC/Images/BITSSlogo.png}
 \logo{\pgfuseimage{university-logo}}

% Delete this, if you do not want the table of contents to pop up at
% the beginning of each subsection:
%\AtBeginSubsection[]
%{
%  \begin{frame}<beamer>{Outline}
%   \tableofcontents[currentsection,currentsubsection]
%  \end{frame}
%}


% If you wish to uncover everything in a step-wise fashion, uncomment
% the following command: 

%\beamerdefaultoverlayspecification{<+->}

\begin{document}

% Since this a solution template for a generic talk, very little can
% be said about how it should be structured. However, the talk length
% of between 15min and 45min and the theme suggest that you stick to
% the following rules:  

% - Exactly two or three sections (other than the summary).
% - At *most* three subsections per section.
% - Talk about 30s to 2min per frame. So there should be between about
%   15 and 30 frames, all told.

\begin{frame}
  \titlepage
\end{frame}

\begin{frame}{Outline}
  \tableofcontents
  % You might wish to add the option [pausesections]
\end{frame}


\section{Introduction}
\begin{frame}{Reproducibility \& Transparency}
\begin{itemize}
\item What are problems associated with reproducibility?
\item What are solutions to these problems?
\item What are practical tools to implement these solutions?
\end{itemize}
\end{frame}
%%%%%%%%%%%%%%%%%%%%%%%%%%%%%%%%%%%%%%%%%%%%%%%%%%%%%%%%%%%%%%%%%%
%%%%%%%%%%%%%%%%%%%%%%%%%%%%%%%%%%%%%%%%%%%%%%%%%%%%%%%%%%%%%%%%%%%%
\section{Replication}
\begin{frame}{Replication}
\begin{enumerate}[<.->]
 \item The Problem	(JMCB Project)
 \item Project Protocol, Reporting Standards
 \item Organizing Workflow
 \item Code \& Data Sharing
\end{enumerate}
\end{frame}

\subsection*{Project Protocol, Reporting Standards}
\begin{frame}[<.->]{Project Protocol, Reporting Standards}
 Make sure you report everything another researcher would need to replicate your research.
 \begin{itemize}
 \item Find the appropriate reporting standard for your field and follow it: \url{http://www.equator-network.org/}
\item Report the nuts and bolts of the project implementation in a detailed protocol: \url{http://www.spirit-statement.org}
\end{itemize}
\end{frame}

 \subsection*{Workflow}
 \begin{frame}{Workflow}
``Reproducibility is just collaboration with people you don't know,
including yourself next week''

---Philip Stark, UC Berkeley Statistics
\end{frame}
\begin{frame}{Workflow}

 Practical coding and organizational suggestions
 \begin{itemize}
 \item Long (2008) \textit{The Workflow of Data Analysis Using Stata}
 \begin{itemize}
 	\item Making any changes to a file that has been posted/shared means it gets a new name.
 	\item Use version commands to ensure others get same results.
  \end{itemize}
 \item Literate programming (extensive commenting, making the aim of code reading by a human)
 \item R Markdown, integration of analysis and output.
\end{itemize}
\end{frame}

\subsection*{Data Sharing}
\begin{frame}{Data Sharing}
Post your code and your data in a trusted public repository.
\begin{itemize}[<.->]
\item
Find the appropriate repository: \url{http://www.re3data.org/}
\item
Repositories will last longer than your own website.
\item
Repositories are more easily searchable by other researchers.
\item
Repositories will store your data in a non-proprietary format that won't become obsolete.
\end{itemize}
\end{frame}


\end{document}


